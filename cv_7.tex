%----------------------------------------------------------------------------------------
%	PACKAGES AND OTHER DOCUMENT CONFIGURATIONS
%----------------------------------------------------------------------------------------

\documentclass[11pt,a4paper,sans]{moderncv} % Font sizes: 10, 11, or 12; paper sizes: a4paper, letterpaper, a5paper, legalpaper, executivepaper or landscape; font families: sans or roman

\moderncvstyle{classic} % CV theme - options include: 'casual' (default), 'classic', 'oldstyle' and 'banking'
\moderncvcolor{blue} % CV color - options include: 'blue' (default), 'orange', 'green', 'red', 'purple', 'grey' and 'black'

\usepackage[utf8]{inputenc}

\usepackage[scale=0.75]{geometry} % Reduce document margins
%\setlength{\hintscolumnwidth}{3cm} % Uncomment to change the width of the dates column
%\setlength{\makecvtitlenamewidth}{10cm} % For the 'classic' style, uncomment to adjust the width of the space allocated to your name

%----------------------------------------------------------------------------------------
%	NAME AND CONTACT INFORMATION SECTION
%----------------------------------------------------------------------------------------

\firstname{Andreas} % Your first name
\familyname{Happe} % Your last name

% All information in this block is optional, comment out any lines you don't need
\title{Curriculum Vitae}
\address{Franzensgasse 10/13}{1050 Vienna, Austria}
\mobile{+43 676 3355006}
\email{andreas@offensive.one}
\extrainfo{github.com/andreashappe}
\photo[70pt][0.1pt]{pictures/picture} % The first bracket is the picture height, the second is the thickness of the frame around the picture (0pt for no frame)
\quote{}

%----------------------------------------------------------------------------------------

\begin{document}

\makecvtitle % Print the CV title

\section{Currently Ongoing}

\cventry{since 2022}{PhD in Science}{Security Impact of Machine Learning}{TU Wien}{}{}
\cventry{since 2021}{Security Freelancer}{Supporting Application Security}{}{}{Helping companies improve their security posture and secure software development practise, ``left-shifting'' security.}

\section{Employment History}

\cventry{2019--2022}{Lecturer}{\textsc{FH Technikum Wien, Vienna}}{}{}{Web Security, Secure Operating Systems, Web Application Security\newline Supervised \textasciitilde10 master student theses, most completed with distinction.}

\cventry{2012--2022}{Senior Security Consultant}{\textsc{CoreTEC GmbH, Vienna}}{}{}{Security-Assessments, Penetration-Tests and Secure Software Development Training}

\cventry{2015--2018}{Engineer}{\textsc{Austrian Institute of Technology}}{}{}{Design, Implementation and Maintenance of privacy-preserving multi-cloud storage, identity management and data-sharing systems.\newline Projects: Credential (Horizon 2020), Prismacloud (Horizon 2020).}

\cventry{2006--2015}{Software Engineering Contractor}{\textsc{Austrian Institute of Technology}}{}{}{2012--2015: Design and of secure multi-cloud storage systems\newline 2006--2012: Design and Development of a Quantum Key-Distribution system (SECOQC).}

\cventry{2009--2015}{Ruby on Rails Freelancer}{}{}{}{Design, Development and Maintenance of Ruby on Rails-based web applications.}

\cventry{2007--2009}{CTO}{\textsc{BlackWhale GmbH}}{}{}{Startup working on web-based work-flow solutions.}

\cventry{2001--2007}{System Administrator}{\textsc{Infotech GmbH}}{}{}{Linux and Microsoft Windows systems.}

\section{Reference Projects}

\cventry{2022, 2023}{TTTech Industrial Automation AG}{}{}{}{Cybersecurity Schulung für Entwickler}

\section{Technical Skills}

\cvitem{Security Engineering}{Design, Execution and Documentation of Penetration-Tests.\newline Primary Focus upon Web-Applications as well as Android/iOS Mobile Applications.\newline Design and Execution of training events in the Security Area.}
\cvitem{Secure Software Engineering}{Assessment, Design and Implementation of secure IT-Systems.\newline Review and Improvement of Secure Software Development Lifecycles.\newline Support with automated tooling in the CI/CD/SAST area.\newline Software Development in Compliance with ÖNORM A77.00, ISO27001 and CMMC.}
\cvitem{Software Development}{Procedural, Object-Oriented and Functional Programming Paradigms.\newline Expert level in \textsc{Ruby on Rails}, \textsc{Python}, \textsc{C}, \textsc{Go}, \textsc{Java}\newline
Proficient in \textsc{Scala}, \textsc{JavaScript}, \textsc{Rust}.}

\section{Languages}

\cvitemwithcomment{German}{Native language}{}
\cvitemwithcomment{English}{Full professional proficiency}{}

\section{Certifications}

\cvitem{2023}{TCM Security: Practical Network Penetration Tester (PNPT)}

\cvitem{2020--2022}{NIS-G Auditor für Kritische Infrastruktur}

\cvitem{2015}{Offensive Security Certified Professional (OSCP)}

\section{Standardization Work}

\cvitem{2023}{OWASP ``Proactive Security Controls''\newline\footnotesize{Co-Leader}}
\cvitem{2017}{OWASP MSTG -- ``Mobile Security Testing Guide''\newline\footnotesize{Top Contributor}}
\cvitem{2016--2019}{ÖNorm A77.00 -- ``Sichere Webapplikationen''\newline\footnotesize{Austrian Standard on Development and Maintenance of Secure Web Applications}}

\section{Additional Security Involvement}

\cventry{since 2023}{OWASP Co-Leader}{\textsc{OWASP Proactive Security Controls}}{}{}{}

\cventry{since 2019}{OWASP Leader}{\textsc{Chapter Vienna}}{}{}{}

\cvitem{2019}{Author \href{https://amzn.to/2ZqysuD}{Einführung in die Web Application Security}}

\cvitem{2019}{We Are Developers -- Sounding Board Security}

\cvitem{2019}{NATO Locked Shields, Partner Event (2nd place)\newline Teamlead Web-Security, Team FH/Technikum Wien}

\newpage

\section{Formal Education}

\cventry{2006--2009}{DI/Master of Science}{Software Engineering \& Internet Computing}{TU Wien}{}{}
\cventry{2002--2006}{Bakk.\ techn.}{Software \& Information Engineering}{TU Wien}{}{}
\cventry{1996--2001}{Matura}{EDV und Organisation}{HTBLVA Villach}{}{}


\section{Masters Thesis}

\cvitem{Title}{\emph{Agile Provenance}}
\cvitem{Supervisors}{S. Dustdar, L. Juszczyk, H.-L. Truong}
\cvitem{Description}{Automated transparent provenance gathering and analysis within Ruby on Rails.}

\section{Selection of Noteworthy Research Projects}

\cvitem{2015--2018}{\textsc{Prismacloud}\newline Design, development and maintenance of the PrismaCloud privacy-preserving multi-cloud storage prototype. One of seven projects accepted for the European Union's Horizon 2020 Research Programme.}

\cvitem{2015--2018}{\textsc{Credential}\newline Development of Trust Solutions for untrusted multi-cloud architectures. Another one of the seven projects accepted for the European Union's Horizon 2020 Research Programme.}

\cvitem{2012--2015}{\textsc{Archistar}\newline Design and Development of a Multi-Cloud Storage System utilizing BFT (Byzantine Fault Tolerance) and Secret-Sharing techniques.}

\cvitem{2006--2012}{\textsc{SECOQC}\newline Implementation of the first inter-company quantum key distribution network. I was deeply involved in design and implementation of the networking components  (which were written using Linux, Python, C). After the presentation of the prototype during the SECOQC-Conference of 2009 responsible for maintenance and further feature-work.}


\newpage

\section{Publications -- Security and Machine Learning}

\cvitem{2023}{Getting pwn'd by AI: Penetration Testing with Large Language Models\newline\footnotesize{Andreas Happe, Jürgen Cito\newline Will present at FSE 2023 IVR in San Fransisco, USA}}

\cvitem{2023}{Understanding Hackers’ Work: An Empirical Study of Offensive Security Practitioners\newline\footnotesize{Andreas Happe, Jürgen Cito\newline Will present at FSE 2023 Industrial Track in San Fransisco, USA}}

\section{Publications -- Unikernel}

\cvitem{2017}{Unikernels for Cloud Architectures: How Single Responsibility can Reduce Complexity, Thus Improving Enterprise Cloud Security\newline\footnotesize{Andreas Happe, Bob Duncan, Alfred Bratterud\newline Presented at Complexis 2017 in Porto, Portugal}}
\cvitem{2016}{Enterprise IoT Security and Scalability: How Unikernels can Improve the Status Quo\newline\footnotesize{Bob Duncan, Andreas Happe, Alfred Bratterud\newline IEEE/ACM 9th International Conference on Utility and Cloud Computing\newline 2016 in Shanghai, China}}
\cvitem{2016}{Enhancing Cloud Security and Privacy: Time for a New Approach?\newline\footnotesize{Bob Duncan, Alfred Bratterud, Andreas Happe\newline INTECH 2016 in Dublin, Ireland}}

\section{Publications --- Cloud Storage}

\cvitem{2017}{The Archistar Secret-Sharing Backup Proxy\newline\footnotesize{Andreas Happe, Florian Wohner, Thomas Loruenser\newline SECPID/ARES 2017 in Calabria, Italy}}
\cvitem{2016}{Exchanging Database Writes with modern Crypto\newline\footnotesize{Andreas Happe, Thomas Loruenser\newline Presented at IARIA Cyber 2016 in Venice, Italy}}
\cvitem{2016}{Malicious Clients in Distributed Secret Sharing Based Storage Networks\newline\footnotesize{Andreas Happe, Stephan Krenn, Thomas Loruenser\newline Presented at Secure Protocol Workshop 2016 in Brno, Czech Republic}}
\cvitem{2015}{\textsc{ARCHISTAR}: Towards Secure and Robust Cloud Based Data Sharing\newline \footnotesize{Thomas Loruenser, Andreas Happe, Daniel Slamanig\newline Presented at IEEE CloudCon 2015 in Vancouver, Canada}}

\newpage
\section{Publications --- Quantum Key Distribution}

\cvitem{2013}{New release of an open source QKD software: design and implementation of new algorithms, modularization and integration with IPSec\newline \footnotesize{Andreas Happe, Oliver Maurhart, Christoph Pacher, Thomas Loruenser, Cristina Tamas, Andreas Poppe, Momtchil Peev et al.}}
\cvitem{2012}{Timing synchronization with photon pairs for quantum communications\newline \footnotesize{Andreas Happe, Thomas Loruenser, Andreas Poppe, Momtchil Peev, Florian Hipp, Damian Melniczuk, Pattama Cummon, Pituk Panthong, Paramin Sangwongngam et al.}}
%\cvitem{2012}{Quantum Key Distribution Software maintained by AIT\newline\footnotesize{Andreas Happe, Oliver Maurhart, Christoph Pacher, Thomas Loruenser, Gottfried Lechner, Cristina Tamas, Andreas Poppe, Momtchil Peev et al.}}
\cvitem{2012}{QKD software architecture and system integration with classical communication infrastructure\newline\footnotesize{Oliver Maurhart, Christoph Pacher, Andreas Happe, Thomas Loruenser, Cristina Tamas, Andreas Poppe, Momtchil Peev}}
\cvitem{2009}{The SECOQC quantum key distribution network in Vienna\newline\footnotesize{Andreas Happe, Momtchil Peev, Thomas Loruenser, Thomas Themel, Christoph Pacher, Oliver Maurhart, Andreas Poppe, Anton Zeilinger, Cristina Tamas, Edwin Querasser et al.}}

%----------------------------------------------------------------------------------------

\end{document}
